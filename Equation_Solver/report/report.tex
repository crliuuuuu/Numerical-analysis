\documentclass{ctexart}

\usepackage{ctex}
\usepackage{tikz}
\usetikzlibrary{calc,positioning,shapes.geometric}
\usepackage{url}
\usepackage{graphicx}
\usepackage{float}
\usepackage{xcolor}
\usepackage{color}
\usepackage{amsmath}
\usepackage{amsthm}
\usepackage{amssymb}
\usepackage{mathrsfs}
\usepackage{caption}
\usepackage{subfigure}
\usepackage{framed}
\usepackage{booktabs}
\usepackage{makecell}
\usepackage{geometry}
\usepackage{wrapfig}
\usepackage{abstract}
\usepackage{algorithmicx}
\usepackage[ruled]{algorithm}
\usepackage{algpseudocode}
\usepackage{setspace}
\usepackage{bm}
\usepackage{cite}
\usepackage{array}
\usepackage{textcomp}
\usepackage{listings}

\definecolor{shadecolor}{rgb}{0.93,0.93,0.93}
\usepackage{geometry}
\geometry{right=2.5cm,left=2.5cm}

\newtheorem{theorem}{定理}

\pagenumbering{arabic}

\begin{document}
\begin{sloppypar}
\title{\vspace{-3cm} \textbf{数值分析第一章编程作业说明}}
\author{刘陈若\;$3200104872$\\信息与计算科学2001}
\date{}

\maketitle

\section*{程序编译和运行说明}
本次项目作业采用Makefile文件对编译进行统一管理。具体地,在Makefile所在目录下输入\verb|make|
即可完成编译,得到problem\verb|B|-problem\verb|F|的可执行文件\verb|B|,\verb|C|,\verb|D|,\verb|E|,\verb|F|。分别对其进行运行,即可得到各小题的输出结果。

\section*{程序运行结果及简要分析}
\subsection*{Problem A}
问题A没有数值输出,其具体的要求都已经全部在头文件\verb|equationsolver.h|中得到实现。值得注意的是,\verb|equationsolver.h|中还通过构建基类Function完成对用户函数输入的规范化。

\subsection*{Problem B}
问题B需要对四个函数使用二分法分别求根,其具体输出如下:
\begin{shaded}
\begin{verbatim}
root of function 1 is r1 =  0.860334, f(r) = 0
root of function 2 is r2 =  0.641186, f(r) = 0
root of function 3 is r3 =  1.82938, f(r) = 0
root of function 4 is r4 =  0.117877, f(r) = -1.21841e+16
\end{verbatim}
\end{shaded}

可以看出,函数4运用二分法的结果非常糟糕,计算得到的根实际的函数值趋向无穷大。对此的解释为:函数4在区间$[0,4]$中不是连续的,因此不符合二分法对函数的连续性要求,实际求得的根将会接近于奇点。

\subsection*{Problem C}
问题C需要对方程$x = tanx$运用牛顿法求出其在4.5以及7.7附近的根,其具体输出如下:
\begin{shaded}
\begin{verbatim}
root near 4.5 is r1 =  4.49341, f(r) = -8.88178e-16
root near 7.7 is r2 =  7.72525, f(r) = 2.30926e-14
\end{verbatim}
\end{shaded}
可以发现牛顿法在本题中的运用是较为理想的。

\subsection*{Problem D}
问题D首先对三个函数用割线法求根。其次,对于方程1,将其初始值的$x_1$更改为$4.5\pi$;对于方程2,将其初始值$x_1$更改为5.4;对于方程3,将其初始值更改为0.35和0.45,可以得到不同的求根结果。其具体输出如下:
\begin{shaded}
\begin{verbatim}
root of function 1 is r1 =  3.14159, f(r) = -1.11022e-16
root of function 1 with different initial values is r1 =  15.708, f(r) = -1.11022e-16
root of function 2 is r2 =  1.30633, f(r) = -4.44089e-16
root of function 2 with different initial values is r2 =  0.918663, f(r) = 1.19631
root of function 3 is r3 =  -0.188685, f(r) = 0
root of function 3 with different initial values is r3 =  0.451543, f(r) = 2.22045e-16
\end{verbatim}
\end{shaded}
可以看出,割线法求根的精确度和初始值$x_0$,$x_1$更加密切相关。当初始值较接近真实根的时候,结果较为准确;当初始值$x_0$,$x_1$位于不同位置或者间距过大,可能会收敛到方程其它的根;当$x_0$,$x_1$中含有方程的奇点时,割线法可能会在较长时间内无法收敛或者收敛停止到非真实根处。

\subsection*{Problem E}
问题E需要对水槽中水的深度(在本说明中理解为$r-h$)分别使用二分法,牛顿法和割线法进行精确到0.01的估计。其具体输出如下:
\begin{shaded}
\begin{verbatim}
root using Bisection Method is h =  0.16875, f(h) = 0.0509495 and the depth is: 0.83125
root using Newton Method is h =  0.166166, f(h) = 0 and the depth is: 0.833834
root using Secant Method is h =  0.166165, f(h) = -2.13731e-05 and the depth is:
0.833835
\end{verbatim}
\end{shaded}
输出的结果都已经满足题目要求了。但需要说明的是,为了达到0.01的精度,实际上在三种方法中设置的停止条件和初始条件要更为严苛一些。

\subsection*{Problem E}
\subsubsection*{(a)}
本题中需要对题目给出的方程中运用牛顿法求得最大可能角度$\alpha$的值,并证实$\alpha \approx33°$。其具体输出如下:
\begin{shaded}
\begin{verbatim}
root using Newton Method in (a) is alpha = 32.9722°, f(alpha) = 0
\end{verbatim}
\end{shaded}
\subsubsection*{(b)}
本题中需要对修改参数$D=30$后的方程中运用牛顿法求得最大可能角度$\alpha$的值。其具体输出如下:
\begin{shaded}
\begin{verbatim}
root using Newton Method in (b) is alpha = 33.1689°, f(alpha) = 0
\end{verbatim}
\end{shaded}
\subsubsection*{(c)}
本题中首先需要对题目给出的方程中运用割线法求得最大可能角度$\alpha$的值。其次,通过改变远离$33°$的初始值(接近$180°$),可以得到不同的结果。其具体输出如下:
\begin{shaded}
\begin{verbatim}
root using Secand Method in (c) is alpha = 32.9722°, f(alpha) = 0
root using Secand Method with different initial value in (c) is alpha = 168.5°, 
f(alpha) = -3.55271e-15
\end{verbatim}
\end{shaded}
可以发现,当初始值远离$33°$时,求得的根发生了变化,且精度也略有区别。这是因为,通过软件$GeoGebra$画图可得,函数的根并不唯一,因此当初始值发生改变,收敛的根也会发生改变,且根据问题$D$中的分析,当初始值改变的幅度不同,得到的根的位置的误差也会不同。

\end{sloppypar}
\end{document}
